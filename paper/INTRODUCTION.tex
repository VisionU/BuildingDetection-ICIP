\section{INTRODUCTION}
\label{sec:intro}
	Extracting building rooftops automatically from very-high-resolution (VHR) satellite image has been an intensive research topic in the fields of remote sensing, compute vision  and computer graphics  over the last few decades. This is because building rooftops plays a critical role in  a diverse range of applications, such as  urban monitoring, disaster assessment, 3D city modelling, and military reconnaissance. Though a series of approaches has been developed for rooftop extraction from optical remote sensing image, it is still remains a challenging task to develop generic and robust algorithms. One reason is that the images used often differ in terms of lighting conditions, quality.  Another reason is that buildings may have diverse shapes and colors and as such can be easily confused with similar objects such as cars, roads, and courtyards. In the literature, large amounts of efforts achieve good performance in detecting buildings with special color, shapes, or textures.
	
	One popular way of extracting buildings is exploring their shapes. It is observed that rooftops have more regular shapes,
which usually are rectangular or combinations of several rectangles. A dozen years ago, Noronha and Nevatia \cite{noronha2001detection} designed a system that detects and constructs 3D models for rectilinear buildings from multiple aerial images. Hypotheses for rectangular rooftop were generated by grouping lines at first, then verified by searching for presence of predicted walls and shadows. Nosrati and Saeedi \cite{nosrati2009novel} proposed approach uses edge definitions and their relationships with each other to create a set of potential vertices. Polygonal rooftops correspond to closed loops in a graph which represents the relationship between potential vertices in an efficient way. In \cite{izadi2012three}, Izadi and Saeedi used a graph-based search to establish a set of rooftop hypotheses through examining the relationships of lines and line intersections. Cui \textit{et al.} \cite{cui2012complex} used the Hough transform (HT) to extract buildings, but HT had notable drawbacks in parameter tuning and time complexity. Cote and Saeedi \cite{cote2013automatic} generated rooftop outline from selected corners in  multiple color and color-invariance spaces, further refined to fit the best possible boundaries through level-set curve evolution. In \cite{wang2015efficient}, Wang \textit{et al.}  presented a graph search-based perceptual grouping approach to hierarchically group line segments detected by EDLines \cite{akinlar2011edlines} into candidate rectangular buildings, computation complexity of the approach was reduced dramatically compared to \cite{noronha2001detection} \cite{izadi2012three} \cite{cote2013automatic} \cite{mayunga2007semi}. However, geometric primitives based methods suffer from three serious shortcomings. The first one is that they lack the ability of detecting arbitrarily shaped building rooftop. The second one is that they fail to  extract credible geometric features in buildings with inhomogeneous color distribution or low contrast with surroundings. The third one is that it is hardly possible to apply to process large-scale because of its high computational-complexity.	 
	
	Several studies reported that buildings are often composed of homogeneous regions with similar color or texture nearby shadows in remote sensing images. Spectral features is a distinctive feature for object detection, for instance, shadows are commonly dark grey or black, vegetations are usually green or yellow with particular textures, and main roads are dim gray with different road marks in most case. According to these prior knowledge mentioned above, Ghaffarian \textit{et al.}  \cite{ghaffarian2014automaticPFICA} proposed an PFastICA (Purposive Fast Independent Component Analysis) technique to separate building area from remote sensing image. However, Ghaffarian's approach fails to detect the buildings with significantly different coloured rooftops. In \cite{chen2014shadow}, Chen \textit{et al.} proposed a supervised learning based building detection framework. At first, the image was divided into super-pixels using the SLIC \cite{achanta2012slic} algorithm, then shadows patches are recognized using the LDA color feature and the SVM classifier. The rough segmentation of buildings is employed by an adaptive regional growth algorithm that considers the spatial relationship between shadows and buildings. Finally, buildings are segmented accurately using a level set model. In \cite{ghaffarian2014automaticsupervised}, illumination direction and shadow area information of training samples were collected firstly, and then a improved parallelepiped classification method was applied to classify the image pixels into building and non-building areas. Dornaika \textit{et al.} \cite{dornaika2015object} proposed a similar framework which combining a powerful segmentation algorithm with a hybrid descriptor composed by color histograms and several Local Binary Patterns based descriptors. Each region generated by statistical region merging (SRM) algorithm was classified using machine learning tools and a gallery of training descriptors. However, a major problem of these methods is that they have no capability to detect building rooftop with significantly varying illumination in different parts.	
	
	Previously published studies present synthetic approaches with fusion of shadow, spectral, and structure to extract building profiles from aerial images. Baluyan \textit{et al.} \cite{baluyan2013novel} proposed a method based on both spectral and spatial features extracted from the images. Firstly, k-means clustering is used to segment the image into a set of rooftop candidates. Secondly, the SVM classifier is trained to distinguish between rooftop regions and nonrooftop regions using extracted features in dataset. Finally, "histogram method" was devised to detect missed rooftops in previous step. In \cite{ngoautomatic}, Ngo \textit{et al.} presented a novel approach for the automated detection of rectangular buildings. At first step, the image is decomposed into small homogeneous regions as candidates. Next step, a merging process is then performed over regions having similar spectral traits to produce rectilinear building region in accordance with position of shadows. Li \textit{et al.} \cite{li2015robust} proposed a higher order conditional random field (CRF) based method, which incorporates both pixel-and segment-level information for the segmentation of rooftops. They claimed that the proposed model outperforms the state-of-the-art methods experiment at rooftops with complex structures and sizes. 	
	
	Based on those observations, all of the  above mentioned approaches deal with building rooftop detection in small-scale image with about dozens of buildings. It is not effective to process very large-scale image using proposed method. Some related works aim to extract built-up area in large-scale image is reviewed as following. Liu \textit{et al.} \cite{liu2013perception} proposed a perceptual built-up index associates with corners to recognize built-up areas from high-resolution remote sensing images. In \cite{tao2013unsupervised},
Tao  \textit{et al.} argued that built-up areas can be discriminated by frequently recurring appearance patterns or repeated textures. 
 First, they extracted a large set of corners from each input image by an improved Harris corner detector. Afterward, they incorporated the extracted corners into a likelihood function to locate candidate regions in each input image. Secondly, the candidate regions is solved by spectrum clustering and graph cuts. Li \textit{et al.} \cite{li2015built} proposed a supervised approach using multikernel learning, multifield integrating, and multihypothesis voting to address the determination of the block size and the pursuit of the pixel-level result. We propose a novel method based on coarse to fine strategy to extract building rooftop from large-scale image, the  illumination robust over-segmentation method is proposed to segment image into a set of patches. 
	
	

